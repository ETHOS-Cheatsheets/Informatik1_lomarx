\documentclass[a3paper, 10pt, landscape]{scrartcl}

\usepackage[german]{babel} %choose your language
\usepackage[landscape, margin=1cm]{geometry}
\usepackage[utf8]{inputenc}
\usepackage[dvipsnames]{xcolor}
\usepackage{amscd, amsmath, amssymb, blindtext, empheq, enumitem, multicol, parskip}
\usepackage{graphicx}
\usepackage{tikz}
\usepackage{esint}
\usepackage{wrapfig}
\usepackage{setspace}
\usepackage{trfsigns}
\usepackage{mathtools}
\usepackage{cmbright,bm}
\usepackage{esvect}
\usepackage{makecell}

\RedeclareSectionCommands[
  %runin=false,
  afterindent=false,
  beforeskip=.1\baselineskip,
  afterskip=.1\baselineskip]{section}
\RedeclareSectionCommands[
  %runin=false,
  afterindent=false,
  beforeskip=.1\baselineskip,
  afterskip=.1\baselineskip]{subsection}
\RedeclareSectionCommands[
  %runin=false,
  afterindent=false,
  beforeskip=.1\baselineskip,
  afterskip=.1\baselineskip]{subsubsection}
%\titlespacing{\section}{0pt}{*0}{*0}
%\titlespacing{\subsection}{0pt}{*0}{*0}
%\titlespacing{\subsubsection}{0pt}{*0}{*0}

\parindent 0pt
\pagestyle{empty}
\setlength{\unitlength}{1cm}
\setlist{leftmargin = *}

\setlength{\parskip}{0cm}
\setlength{\parindent}{0em}

\def\StyleColor{Apricot}

\DeclareMathOperator{\rot}{rot}
\DeclareMathOperator{\divg}{div}

\usepackage{color}
\definecolor{section}{RGB}{175, 60, 230}
\definecolor{subsection}{RGB}{211, 153, 230}
\definecolor{subsubsection}{RGB}{238, 218, 229}
\definecolor{titletext}{RGB}{0,0,0}
\definecolor{formula}{RGB}{220,230,255}

% section color box
\setkomafont{section}{\mysection}
\newcommand{\mysection}[1]{
    \Large\bfseries
    \setlength{\fboxsep}{0cm}
    \colorbox{section}{
        \begin{minipage}{\linewidth-0.11cm}
            \vspace*{1pt}
            \leftskip2pt 
            \rightskip\leftskip
            {\color{titletext} #1}
            \vspace*{1pt}
        \end{minipage}
    }}
%subsection color box
\setkomafont{subsection}{\mysubsection}
\newcommand{\mysubsection}[1]{
    \normalsize \bfseries
    \setlength{\fboxsep}{0cm}
    \colorbox{subsection}{
        \begin{minipage}{\linewidth}
            \vspace*{1pt}
            \leftskip2pt 
            \rightskip\leftskip 
            {\color{titletext} #1}
            \vspace*{1pt}
        \end{minipage}
    }}
    
%subsubsection color box
\setkomafont{subsubsection}{\mysubsubsection}
\newcommand{\mysubsubsection}[1]{
    \normalsize \bfseries
    \setlength{\fboxsep}{0cm}
    \colorbox{subsubsection}{
        \begin{minipage}{\linewidth}
            \vspace*{1pt}
            \leftskip2pt 
            \rightskip\leftskip
            {\color{titletext} #1}
            \vspace*{1pt}
        \end{minipage}
    }}    


\newcommand{\dis}[1]{\hspace{#1cm}}
    
% equation box        
\newcommand{\eqbox}[1]{\fcolorbox{section}{formula}{\hspace{0.5em}$\displaystyle#1$\hspace{0.5em}}}

%macro for vectors 
%\newcommand{\vect}[1]{\vec{#1}}
\newcommand{\vect}[1]{\boldsymbol{#1}}

%code snippet
\newcommand{\code}[1]{\texttt{#1}}

%\renewcommand{\familydefault}{cmss}
\DeclareSymbolFont{letters}{OML}{ztmcm}{m}{it}
\DeclareSymbolFontAlphabet{\mathnormal}{letters}

%\everymath{\displaystyle}  %bigger equations
\begin{document}
%\setcounter{secnumdepth}{0} %no enumeration of sections
	\begin{multicols*}{3}
	
	
%-------------------------------------------------------
\section*{Disclaimer}
    \vspace{0.2cm}
    Diese Zusammenfassung wurde zur Vorlesung "`Informatik I"' von Dr. Malte Schwerhoff und Dr. Hermann Lehner (FS20) erstellt. Die Zusammenfassung soll und darf gerne modifiziert werden (Latex-Files liegen bei), und soll dann auch weiterhin anderen Studenten zur Verfügung stehen. \\
			
    Für Korrektheit und Vollständigkeit ist keine Gewähr.\\
			
    Josephine Loehle und Leo Landolt, 21. Mai 2021

    Angepasst von Loa Marx, 21. Dezember 2023
    \vfill\null
\pagebreak
	
\section{Grundlagen}

	\subsection{Typen}
	
		\subsubsection{Overview}
		\vspace{0.1cm}
		
		\begin{tabular} {l l c l}
			\textbf{Typ} & \textbf{Was} & \textbf{Werte} & \textbf{Bits} \\
			\hline
			double & präzisere Reelle Zahlen & $1.7*10^{\pm308}$ & 64 \\
			float & Reelle Zahlen & $3.4*10^{\pm38}$ & 32\\
			unsigned int & Natürliche Zahlen & 0 bis $ 2^{32}-1$ & 32\\
			int & Ganze Zahlen & $-2^{31}$ bis $2^{31}-1$ & 32\\
			
			bool & Wahrheitswerte & true (1) oder false (0) & 8\\
			\hline
			
			char & Zeichen & & 8 \\
			
			
			
			\end{tabular}\\
			
			Bei einem Konflikt wird in den allgemeineren $\uparrow$ Zahlentyp konvertiert.\\
			Bsp.: \code{int a + float b = float c}
			\vspace{0.1cm}
		
		\subsection{Fliesskommazahlen}
		
		\subsubsection{Fliesskommazahlensysteme}
		\vspace{0.1cm}
		Durch 4 natürliche Zahlen definiert: F($\beta$, p, e$_{min}$, e$_{max}$) \\
		\begin{tabular}{l l}
			\textbf{$\beta\geq2$} & Die Basis \\
			\textbf{p $\geq1$} & Die Präzision (= Stellenanzahl, Mantisse) \\
			\textbf{e$_{min}$} & Der kleinste Exponent \\
			\textbf{e$_{max}$} & Der grösste Exponent
		\end{tabular}
		
		\subsubsection{Normalisierte Darstellung}
		
		\centerline{$\pm d_0$.$d_1$...$d_{p-1}$*$\beta^e$ = F*($\beta$, p, e$_{min}$, e$_{max}$)}
		\begin{itemize}
			\item Eine Ziffer (!= 0) vor dem Komma, der Rest dahinter
			\item Normalisierte Darstellung ist eindeutig
			\item Die Zahl 0 und alle Zahlen kleiner als $\beta^{e_{min}}$ haben keine normalisierte Darstellung
			\item Zum Runden eine Stelle nach erwünschter Präzision anschauen: falls 0 einfach ignorieren, falls 1 die Stelle davor +1 rechnen
		\end{itemize}
		
		\subsubsection{IEEE Standard 754}
		\begin{tabular}{l l}
			\textbf{float} & 1 Bit für das Vorzeichen \\
			& 23 Bit für den Signifikanden (= Bits nach dem Komma) \\
			& 8 Bit für den Exponenten (254 Exponenten, 2 Spezialwerte) \\
			\hline
			& insgesamt 32 Bit \\
			& F*(2, 24, -126, 127) \\
			\hline
			\textbf{double} & 1 Bit für das Vorzeichen \\
			& 52 Bit für den Signifikanden (= Mantisse -Bits) \\
			& 11 Bit für den Exponenten (2046 Exponenten, 2 Spezialwerte) \\
			\hline
			&insgesamt 64 Bit \\
			& F*(2, 53, -1022, 1023) \\
			
			\hline\hline 
			
			\textbf{$\pm$0} &Mantisse-Bits sind 0 \\
			& Exponenten-Bits sind 0 \\
			\hline
			\textbf{$\pm \infty$} & Mantisse-Bits sind 0 \\
			& Exponenten-Bits sind alles 1 \\
			\hline
			\textbf{nan} & Mantisse-Bits $>$ 0 \\
			& Exponenten-Bits sind alles 1
			
			
		\end{tabular}
		
		
		\subsubsection{Float und Double}
		\begin{tabular}{l l}
			\textbf{float} & 7 Stellen \\
			& Exponent bis $\pm38$ \\
			\textbf{double} & 15 Stellen \\
			& Exponent bis $\pm308$ \\
		\end{tabular}
		
		\subsubsection{Fliesskomma-Richtlinien}
		\begin{itemize}
			\item Teste keine Fliesskommazahlen auf Gleichheit
			\item Addiere keine Fliesskommazahlen sehr unterschiedlicher Grösse
			\item Subtrahiere keine Fliesskommazahlen ähnlicher Grösse
			\item Der Modulo-Operator \code{\%} existiert nicht
			\item Division ist mit Nachkommastellen
		\end{itemize}
		
			
		\subsection{Konstanten}
		\begin{itemize}
			\item const type name
			\item Wert der Variable name darf nicht mehr verändert werden.
		\end{itemize}
		
	\subsection{Werte}
	
		\subsubsection{L-Wert}
		\textbf{Links} vom Zuweisungsoperator. \\
		Speicher, hat eine Adresse. \\
		Kann seinen Wert ändern.
		\subsubsection{R-Wert}
		\textbf{Rechts} vom Zuweisungsoperator. \\
		Kann seinen Wert nicht ändern. \\
		L-Wert kann als R-Wert verwendet werden aber nicht andersherum.
		
	\subsection{Operatoren}
	
	\subsubsection{Overview}
	\begin{tabular}{l c l l l}
	\textbf{Operator} & \textbf{Zeichen} & \textbf{Eingabe} & \textbf{Ausgabe}\\
	\hline
	Arithmetische Operatoren \\
	Multiplikation & \code{*} & R-Werte & R-Wert\\
	Division & \code{/} & R-Werte & R-Wert\\
	Modulo & \code{\%} & R-Werte & R-Wert\\
	Addition & \code{+} & R-Werte & R-Wert\\
	Substraktion & \code{-} & R-Werte & R-Wert\\
	Post-Inkrement/Dekrement & \code{expr++} & L-Wert & R-Wert\\
	Prä-Inkrement/Dekrement & \code{++expr} & L-Wert & L-Wert\\
	
	\hline
	
	Relationale Operatoren \\
	Kleiner als & \code{<} & R-Werte & R-Wert\\
	Grösser gleich & \code{>=} & R-Werte & R-Wert\\
	Gleich  & \code{==} & R-Werte & R-Wert\\
	Ungleich & \code{!=} & R-Werte & R-Wert\\
	
	\hline
	
	Logische Operatoren \\
	And & \code{\&\&} & R-Werte & R-Wert \\
	Or & \code{||} & R-Werte & R-Wert \\
	Not & \code{!} & R-Wert & R-Wert \\
	
	\hline
	
	Zuweisung & \code{=} & R-\&L-Wert & L-Wert \\
	
	Eingabe & \code{>>}  & L-Werte & L-Wert\\
	Ausgabe & \code{<<} & L-\& R-Werte & L-Wert\\
	
	\end{tabular}
	
	\subsubsection{Präzedenzen}
	\begin{enumerate}
		\item Klammern
		\item Not
		\item Arithmetische Operatoren
		\begin{itemize}
			\item Punkt vor Strich
			\item Unäre Operatoren vor binären
		\end{itemize}
		\item Relationale Operatoren
		\item Binäre logische Operatoren
		\begin{itemize}
			\item \code{\&\&} vor \code{||}
		\end{itemize}
	\end{enumerate}
	
	\subsubsection{Zusätzliches}
	\begin{itemize}
		\item (a/b)*b + (a\%b) == a
		\item XOR(a,b): (a $\mid\mid$ b) \&\& !(a \&\&b)
		\item De Morgan: !(a \&\& b) == (!a $\mid\mid$ !b) bzw. !(a $\mid\mid$ b) == (!a \&\& !b)
		\item false \&\& $(\dots) \rightarrow$ false
		\item true $\mid\mid$ $(\dots) \rightarrow$ true

		
	\end{itemize}
	
	\subsection{Zahlensysteme}
	
	\begin{tabular}{l l}
		\textbf{Binär} & Basis ist 2 \\
		\textbf{Dezimal} & Basis ist 10 \\
		\textbf{Hexadezimal} & Basis ist 16 mit [0-9] $\hat{=}$ [0-9] und [10-15] $\hat{=}$ [A-F]
	\end{tabular}
	
	\subsubsection{Umrechnung}
	\begin{tabular}{p{1.3cm} p{11.3cm}}
	Dezimal zu Basis & Teile die Dezimalzahl wiederholt durch die Basis und notiere die Reste. Die Zahl in der neuen Basis besteht aus den Resten in umgekehrter Reihenfolge:\\
        & \makecell[l]{18/2 = 9 R\textcolor{red}{0} $\Rightarrow$ 9/2 = 4 R\textcolor{red}{1} $\Rightarrow$ 4/2 = 2 R\textcolor{red}{0} $\Rightarrow$ 2/2 = 1 R\textcolor{red}{0} $\Rightarrow$ 1/2 = 0 R\textcolor{red}{1} \\ $\Rightarrow$ 10010} \\	 \hline
	Basis zu Dezimal & Basis-Potenzen bilden und aufaddieren:10010 $\Rightarrow$ 1*16 + 0*8 + 0*4 + 1*2 + 0*1 $\Rightarrow$ 18 \\ \hline
        Binär zu Hexa. & Jede HD-Ziffer kann einzeln in eine 4-Stellige Binärzahl umgewandelt werden. In derselben Reihenfolge aneinandergliedern: \textcolor{blue}{1}\textcolor{YellowGreen}{A}\textcolor{red}{3} $\Rightarrow$ \textcolor{blue}{1}\textcolor{YellowGreen}{10}\textcolor{red}{3} $\Rightarrow$ \textcolor{blue}{0001}\textcolor{YellowGreen}{1010}\textcolor{red}{0011} $\Rightarrow$ 110100011\\
	
		
	\end{tabular}
 
        \subsubsection{Binär}
	\begin{tabular}{l l l l l l l l l l l }
        $2^{10}$ & $2^9$ & $2^8$ & $2^7$ & $2^6$ & $2^5$ & $2^4$ & $2^3$ & $2^2$ & $2^1$ & $2^0$ \\
        1024 & 512 & 256 & 128 & 64 & 32 & 16 & 8 & 4 & 2 & 1 \\
        \end{tabular}
		\subsubsection{Hexadezimal}
        \begin{tabular}{l l l l}
        $16^{3}$ & $16^2$ & $16^1$ & $16^0$ \\
        4096 & 256 & 16 & 1
        \end{tabular}

%--------------------------------------------
	
\section{Schleifen}
	
	\subsection{if, else if, else}
	\code{\textcolor{red}{if} (condition) \{ \\
	\phantom{if} statement; \\
	\} 
	\textcolor{blue} {$\backslash \backslash$ statement is executed if condition is true} \\
	\newline
	\textcolor{red}{else if} (condition)\{ \\
	\phantom{else if} statement; \\
	\} \textcolor{blue} {$\backslash \backslash$ like if-loop but only looked at if condition before false} \\
	\newline
	\textcolor{red}{else} \{statement; \\
	\} \textcolor{blue}{$\backslash \backslash$ executed if other conditions are false }
	}
	\subsection{for}
	\code{\textcolor{red}{for} (initialisation; condition; expression) \{ \\
	\phantom{for} statement; \\
	\} \textcolor{blue}{$\backslash \backslash$ (i) initialisation statement is executed. \\
	\phantom{) } $\backslash \backslash$(ii) while condition == true, statement is executed, expression is executed. }
	}
	\subsection{while}
	\code{\textcolor{red}{while} (condition) \{ \\
	\phantom{while} statement; \\
	\} \textcolor{blue}{$\backslash \backslash$ statement is executed while condition == true}
	}
	\subsection{do while}
	\code{\textcolor{red}{do} \{ \\
	\phantom{d} statement;\} \\
	\textcolor{red}{while} (condition); \textcolor{blue}{$\backslash \backslash$ statement is executed once, afterwards while-loop}
	}
	\subsection{switch}
	\code{\textcolor{red}{int} grade;\\
	\textcolor{red}{switch}(grade) \{ \\
	\phantom{a} \textcolor{red}{case} 6: statement1; \textcolor{blue}{$\backslash \backslash$matching case is executed} \\
	\phantom{a} \textcolor{red}{case} 5: statement2; \\
	\phantom{a} \textcolor{red}{case} ...; \\
	\phantom{ab} \textcolor{red}{break}; \textcolor{blue}{$\backslash \backslash$ all statements until break are executed (Durchfallen)} \\
	\phantom{a} \textcolor{red}{case} 3: statement3; \\
	\phantom{a} \textcolor{red}{case} ...; \\
	\phantom{ab} \textcolor{red}{break}; \\
	\textcolor{red}{default}: statement else; \textcolor{blue}{ $\backslash \backslash$ if no case matches}
	}
	
	\subsection{Schleifen umwandeln}
	
	\begin{tabular}{l | l}
	\textbf{Anweisung A} & \textbf{Anweisung B} \\
	\hline
	\code{\textcolor{red}{while} (condition) statement;} & \code{\textcolor{red}{for} ( ; condition; ) statement;} \\
	\hline
	\code{\textcolor{red}{do} statement;}  & \code{statement} \\
	\code{\textcolor{red}{while} (condition);} & \code{\textcolor{red}{while} (condition) statement;} \\
	\hline
	\code{\textcolor{red}{for} (\textcolor{red}{int} i = 0; i $<$ n; i++)\{} & \code{\{\textcolor{red}{int} i = 0;}\\
	\code{\phantom{for }statement;} & \code{while(i $<$ n) \{ statement; i++; \}\}}
	\end{tabular}
	
	
	\subsection{Sprunganweisungen}
	\begin{tabular}{l l}
	\textbf{break;} & Schleife wird sofort beendet. \\
	\textbf{continue;} & Rest des Statements wird übersprungen. \\
	& Man gelangt direkt zur Expression.
		
	\end{tabular}
		
%---------------------------------------------	

\section{Funktionen}
	\subsection{Grundlagen}
	\subsubsection{Wieso Funktionen?}
	Kapseln häufig gebrauchte Funktionalitäten und vermeiden somit Code-Duplizierung. 
	
	\subsubsection{Funktionsdefinition}
	\code{\textcolor{red}{Type} fname (\textcolor{red}{type$_1$} pname$_1$, \textcolor{red}{type$_2$} pname$_2$, ..., \textcolor{red}{type$_N$} pname$_N$)\{ \\
	\phantom{Type }block; \\
	\phantom{Type }return Type; \\
	}	
	\subsubsection{Funktionsaufruf}
	\code{fname(expression$_1$, expression$_2$, ..., expression$_N$)\\ \textcolor{blue}{$\backslash \backslash$ expression$_{1-N}$ müssen in type$_{1-N}$ konvertierbar sein.}
	}
	\subsection{Der Typ void}
	\begin{itemize}
		\item Fundamentaler Type mit leerem Wertebereich
		\item Gibt keinen Wert zurück, sondern hat nur einen Effekt
		\item Kein return nötig, aber möglich
	\end{itemize}
	
	\subsection{Vor- und Nachbedingungen}
	\vspace{0.1cm}
	\textcolor{blue}{$\backslash \backslash$ PRE: Was muss bei Funktionsaufruf gelten? \\
	\phantom{// PRE: }Was ist der Definitionsbereich der Funktion? \\
	\phantom{// PRE: }So schwach wie möglich $\rightarrow$ möglichst grosser Definitionsbereich}\\
	\textcolor{blue}{$\backslash \backslash$ POST: Was gilt nach dem Funktionsaufruf? \\
	\phantom{// POST: }Welchen Wert gibt die Funktion zurück? \\
	\phantom{// POST: }So stark wie möglich $\rightarrow$ möglichst detaillierte Aussage} 	
	
	\subsection{Gültigkeitsbereich einer Funktion}
	\vspace{0.1cm}
	\begin{itemize}
		\item Funktion darf nach Deklaration verwendet werden, auch wenn sie erst später wirklich definiert wird.
		\item Deklaration einer Funktion ist wie Definition nur ohne \{\dots\}.
	\end{itemize}
	
	\subsection{Wiederverwenden von Funktionen}
	\subsubsection{Funktion Auslagern und Inkludieren}
	\vspace{0.1cm}
	\begin{itemize}
		\item Funktionen in eigene Datei schreiben (functions.cpp)
		\item Datei ins Arbeitsverzeichnis der Hauptdatei stecken und durch \code{\#include} "\code{functions.cpp}" \space Funktionen inkludieren.
		\item Nachteil: compiler muss die Funktionsdefinition für jedes Programm neu übersetzen
			$\rightarrow$ Kann bei vielen und grossen Funktionen sehr lange dauern
	\end{itemize}
	
	\subsubsection{Getrennte Übersetzung}
	\vspace{0.1cm}
	\begin{itemize}
		\item Funktionen ohne Main-Funktion kompilieren: functions.cpp $\rightarrow$ functions.o
		\item Deklarationen aller benötigten Symbole in einer Header-Datei functions.h
		\item Header-Datei ins Arbeitsverzeichnis stecken und durch \code{\#include} "\code{functions.h}" \space inkludieren
		\item Vorteil: Quellcode (.cpp) wird nach dem Erzeugen vom Objectcode (.o)  nicht mehr gebraucht $\rightarrow$ Code ist nicht öffentlich
	\end{itemize}
	
	\subsubsection{Namensräume}
	\code{namespace std\{ \\
	\} \textcolor{blue}{$\backslash \backslash$Innerhalb der Klammern kann std:: weggelassen werden}
	}
	

%---------------------------------------------

\section{Referenztypen}

	\subsection{Definition, Initialisierung und Zuweisung}
	\vspace{0.1cm}
	\code{\textcolor{red}{Type}\& alias = expr  \textcolor{blue}{$\backslash \backslash$alias ist ein neuer Name für expr}}
	\vspace{0.1cm}
	\subsection{Call by}
	
	\subsubsection{Reference}
	\vspace{0.1cm}
	\begin{itemize}
		\item Funktionsargumente sind (teilweise) Referenztypen.
		\item Durch den Alias ist es möglich, Funktionsargumente dauerhaft, auch ausserhalb der Funktion zu ändern.
	\end{itemize}
	
	\subsubsection{Value}
	\vspace{0.1cm}
	\begin{itemize}
		\item Argumente haben keine Referenztypen.
		\item Beim Funktionsaufruf werden alle Argumente kopiert, am Ende der Funktion werden alle Kopien wieder gelöscht.
		\item Es ist nicht möglich, Funktionsargumente dauerhaft zu ändern.
	\end{itemize}
	
	\subsection{Const-Referenzen}
	\vspace{0.1cm}
	\code{\textcolor{red}{const Type}\& == (\textcolor{red}{const Type})\& \\}
	Es wird ein Lese-Alias gebildet, durch den der Wert dahinter nicht verändert werden darf. \\
	Funktionsargumente müssen so nicht kopiert werden müssen $\rightarrow$ effizient
	\vspace{3cm}

%--------------------------------------------

\section{Vektoren \& Strings}

	\subsection{Vektoren}
	Dienen zum Speichern gleichartiger Daten in einem zusammenhängenden Speicherlayout. \\
	Benötigt \code{\#include $<$vector$>$}
	
	\subsubsection{Vektorinitialisierung}
	
	\begin{tabular}{l l}
	
	\textbf{std::vector$<$int$>$ vec(n);} & Die n Elemente von vec werden mit \\
	& Nullen initialisiert. \\
	\\
	\textbf{std::vector$<$int$>$ vec(n, x);} & Die n Elemente von vec werden mit \\
	&  x initialisiert. \\
	\\
	\textbf{std::vector$<$int$>$ vec\{a, b, c, d\};} &Der Vektor wird mit einer \\
	& Initialisierungslist initialiseirt. \\
	\\
	\textbf{std::vector$<$int$>$ vec;} & Ein leerer Vektor wird initialisiert.
	\end{tabular}
	
	\subsubsection{Matrixinitialisierung}
	\begin{tabular}{l l}
		
	\textbf{std::vector$<$std::vector$<$int$>>$ mat;} & Eine leere Matrix wird initialisiert.\\
	\\
	\textbf{std::vector$<$std::vector$<$int$>>$ mat =} & Matrix wird mittels Initialisierungs-\\
	\textbf{\{ \{a, b, ....\}, \{...\}, ...\};} & liste initialisiert. \\
	\\
	\textbf{std::vector$<$std::vector$<$int$>>$} & Eine n Mal m Matrix wird\\
	\textbf{mat(n, std::vector$<$int$>$(m))} & initialisiert.
	
	
	\end{tabular}
	
	\subsubsection{Auf Elemente zugreifen}
	\textbf{Vektor}\\
	vec[n] gibt einem das n-te Element eines Vektors. \\
	vec.at(n) prüft zusätzlich noch die Vektorgrenzen \\
	\textbf{Matrix}\\
	m[n$_1$][n$_2$] mit 1. Zeile, 2. Spalte \\
	m.at(n$_1$).at(n$_2$) prüft zusätzlich noch Matrixgrenzen \\
	\textcolor{red}{Achtung: Der Index von Vektoren beginnt bei 0!}
	
	\subsubsection{Vektorfunktionen}
	\begin{tabular}{l l}
		\textbf{.size()} & Gibt Länge des Vektors zurück \\
		\textbf{.push\_back(x)} & Fügt Element x hinten an. \\
		\textbf{.empty()} & Prüft, ob der Vektor leer ist.\\
		\textbf{.clear()} & Löscht den Inhalt.\\

	\end{tabular}
	
	\subsection{Zeichen und Texte}
	
	\subsubsection{Der Typ char}
	\vspace{0.1cm}
	\code{\textcolor{red}{char} c = 'a'; \textcolor{blue}{$\backslash \backslash$ Repräsentiert druckbare Zeichen und Steuerzeichen.}\\}
	
	\subsubsection{Strings}
	\code{std::string s =} "\code{I like Pie}" \space\code{\textcolor{blue}{$\backslash \backslash$ Entspricht Vektor von char-Elementen.}\\
	\phantom{std::string s = II like Pie }\space\textcolor{blue}{$\backslash \backslash$ Benötigt \#include $<$string$>$}}\\
	
	\code{std::string text(n, 'a') \textcolor{blue}{$\backslash \backslash$ Initialisiert String der Länge n voller a.}}\\
	
	\begin{itemize}
		\item Strings können verglichen werden (\code{text1 == text2})
	\item Um auf einzelne Characters zuzugreifen benutzt man die gleiche Schreibweise wie bei Vektoren.
	\item Man kann Texte zusammensetzen (\code{text1 += text2})
	\item Die Grösse eines Textes kann ausgegeben werden mit text.length()
	\end{itemize}
	
	\subsubsection{Ströme}
	\begin{tabular}{l l}
		Konsole & benötigt \code{\#include $<$iostream$>$}\\
		& \code{std::cin >> in;} \\
		& \code{std::cout << out;} \\
		& Nur als Referenz für Funktionsargument\\
		\hline
		Dateien & benötigt \code{\#include $<$fstream$>$}\\
		& \code{std::ifstream in(filename)};\\
		& \code{std::ofstream out(filename)};\\
		& Eingabe prüfen: \code{while (in >> input)}\\
		& Analog dazu sstream für Strings\\
		\hline
		Abstrakt & \code{std::istream/std::ostream}\\
		& \\
		
	\end{tabular}
	
	
	\subsubsection{Der ASCII-Code}
	\vspace{0.1cm}
	Definiert konkrete Konversionsregeln char $\leftrightarrow$ (unsigned) int
	
	
	

%---------------------------------------------

\section{Rekursion}
	
	\subsection{Basics}

	\subsubsection{Base Case}
	\vspace{0.1cm}
	\begin{itemize}
		\item Kleinst mögliches Problem
		\item Abbruchbedingung, die sicher erreicht wird
		\item Sonst unendliche Rekursion $\rightarrow$ verbrennt Zeit und Speicher (Stack-Overflow)
	\end{itemize}
	
	\subsubsection{Der Aufrufstapel}
	\vspace{0.1cm}
	\begin{itemize}
		\item Bei jedem Funktionsaufruf kommt ein "`Auftrag"' auf den Aufrufstapel
		\item Sobald der Base-Case erreicht wird, wird der Stapel von oben nach unten abgearbeitet und die Aufträge gelöscht
	\end{itemize}

\section{Structs \& Classes}

	\subsection{Structs}
	\vspace{0.1cm}
	Um sich einen eigenen Typ zu basteln. Default ist public.
	\subsubsection{Definition}
	\vspace{0.1cm}
	\code{\textcolor{red}{struct} T \{ \textcolor{blue}{$\backslash \backslash$T is der Name des neuen Typs} \\
	\phantom{struct } type$_1$ name$_1$; \\
	\phantom{struct } type$_2$ name$_2$; \textcolor{blue}{$\backslash \backslash$type$_{1-n}$ sind die Typen der Membervariablen.}\\
	\phantom{struct } .... \\
	\phantom{struct } type$_n$ name$_n$; \textcolor{blue}{$\backslash \backslash$name$_{1-n}$ sind die Namen der Membervariablen.}  \\
	\}
	}
	\subsubsection{Member-Zugriff}
	
	\code{expr.name$_k$}
	\begin{itemize}
		\item expr ist vom Struct-Typ T
		\item name ist der Name einer Member-Variable des Typs T
		\item . ist der Member-Zugriff-Operator
	\end{itemize}
	
	\subsubsection{Initialisierung}
	\begin{tabular}{l l}
		\textbf{T t;} & Membervariablen von t vom Typ T werden  \\
		& default-initialisiert (Wert undefiniert) \\
		\textbf{T t = \{x, y, \dots\};} & Membervariablen von t werden mit den Werten \\
		& der Liste initialisiert. \\
	\end{tabular}
	
	\subsection{Classes}
	
	\subsubsection{Überladen von Funktionen}
	Man kann mehrere Funktionen mit dem gleichen Namen haben, solange die Signatur anders ist. \\
	Signatur = Namen, Typen, Anzahl und Reihenfolge der Argumente. \\
	\newline
	\textbf{Operator Overloading:} \code{\textcolor{red}{operator}op} \\
	Erlaubt es, die normalen Operatoren (\code{+,-,*,/}) für Structs und Classes zu definieren. \\
	Operatoren sind dann normal anwendbar (Bsp: \code{x + y})
	
	\subsubsection{Datenkapselung}
	\begin{itemize}
		\item Wir verstecken die Repräsentation und bieten Funktionalität.
		\item Struct: nichts wird versteckt (default: public)
		\item Class: alles wird versteckt (default: private)
	\end{itemize}
	
	\subsubsection{Deklaration}
	\code{\textcolor{red}{class} T \{ \\
	\textcolor{red}{public}: \textcolor{blue}{$\backslash$$\backslash$ Memberfunktionen der Klasse, auf die man Zugriff haben soll.} \\
	\textcolor{red}{private}: \textcolor{blue}{$\backslash$$\backslash$ Membervariablen, auf die man kein Zugriff haben soll} \\
	\}; \\}
	\newline
	\textbf{Out-of-class-Memberdefinition} benötigt ein \code{T::}
	
	\subsubsection{Konstruktoren}
	\vspace{0.1cm}
	\begin{itemize}
		\item Sind spezielle Memberfunktionen einer Klasse, die den Namen der Klasse tragen
		\item Können überladen werden
		\item Werden bei der Variablendeklaration aufgerufen
		\item Gibt es keine Variablen bei Deklaration wird der Default-Konstruktor aufgerufen\\ $\rightarrow$ wird automatisch erzeugt wenn keiner definiert \\
	\end{itemize}
	\code{T::T(\textcolor{red}{type$_1$} x, \textcolor{red}{type$_2$} y) : name$_1$ (x), name$_2$ (y) \{ \\
	\phantom{T::T}\textcolor{blue}{$\backslash$$\backslash$ Zusätzliche Bedingungen} \\
	\}
	}\\\\
        Aufruf des Konstruktors:
        %\vspace{0.1cm}
	\begin{itemize}
	\item \code{T t = T(x, y);}
        \item \code{T t(x, y);}
        \end{itemize}
	\subsubsection{this und \code{->}}
	\begin{itemize}
		\item this stellt einen Zeiger auf die momentane Instanz dar
		\item \code{->} dereferenziert einen Zeiger und führt den Punktoperator aus
	\end{itemize}
	


%------------------------------------------

\section{Dynamische Datenstrukturen}

	\subsection{Arrays}
	\subsubsection{Der new-Ausdruck}
	\code{p = \textcolor{red}{new} Type[n]}
	\begin{itemize}
		\item Type ist der zugrundeliegende Typ vom Array
		\item n it die Grösse des zusammenhängenden Speicherplatzes
		\item Der Speicher bleibt reserviert bis man ihn explizit freigibt.
		\item p ist ein Zeiger auf die Startadresse des Speicherbereichs \\
	\end{itemize}
	
	\code{p = \textcolor{red}{new} Type(Konstruktorargumente)}
	\begin{itemize}
		\item Speicher für ein neues Objekt vom Typ T wird alloziiert
		\item p ist ein Zeiger auf die Adresse des Objekts
	\end{itemize}
	
	\subsubsection{Zeiger-Typen}
	\begin{tabular}{l l}
		\textbf{Type* p} & Zeiger auf den Typ Type \\
		\textbf{p = \textcolor{red}{new}...} & Adresse eines neuen Objektes. \\
		\textbf{p = \textcolor{red}{\&}expr} & Adresse eines bereits bestehenden Objektes des Typen Type.
	\end{tabular}
	
	\subsubsection{Dereferenz-Operator}
	\code{expr = \textcolor{red}{*}p}
	\begin{itemize}
		\item p ist ein Zeiger
		\item Gibt den Wert hinter der Adresse p zurück
		\item Adress- und Dereferenzoperator sind inverse Funktionen
	\end{itemize}
	
	\subsubsection{Null-Zeiger}
	\begin{itemize}
		\item Zeigerwert, der angibt, dass auf kein Objekt gezeigt wird
		\item Repräsentiert durch \textcolor{red}{nullptr}
	\end{itemize}
	
	\subsubsection{Zeigerarithmetik}
	\begin{tabular}{l l}
	\textbf{Zeiger plus int} & p + i zeigt auf das i-te Element des Arrays. \\
	\textbf{Zeigersubstraktion} & Differenz beschreibt, wie weit die Elemente \\
	& voneinander entfernt liegen.
	\end{tabular}
	
	\subsubsection{Auf Zeiger zugreifen}
	\textbf{Sequentielle Iteration:} \\
	\code{\textcolor{red}{for} (\textcolor{red}{char}* it = p; it != p + el; ++it)\{ \textcolor{blue}{$\backslash$$\backslash$ el = Anzahl Elemente im Array} \\
	\phantom{for } statement; \\
	\} \\}
	\newline
	\textbf{Wahlfreier Zugriff:} \code{p[i] == *(p + i)}
	
	\subsubsection{Statische Arrays}
	\code{p = \textcolor{red}{int} a[n]}
	\begin{itemize}
		\item Wie Array nur dass die Grösse nicht verändert werden kann.
		\item Vektoren $>$ Dynamische Arrays $>$ Statische Arrays
	\end{itemize}
	
	\subsubsection{Arrays in Funktionen}
	\begin{tabular}{l l}
		\textbf{Konvention} & Array wird durch 2 Zeiger beschrieben. \\
		\textbf{begin} & Zeiger auf das erste Element \\
		\textbf{end} & Zeiger hinter das letzte Element \\
		\textbf{Array leer} & wenn begin == end \\ 
	\end{tabular}
	
	\subsubsection{Const und Zeiger}
	\begin{tabular}{l l}
		\textbf{int const p1} & p1 ist ein const int \\
		\textbf{int const* p2} & p2 ist ein Zeiger auf einen const int \\
		\textbf{int* const p3} & p3 ist ein const Zeiger auf einen int \\
		\textbf{int const* const p4} & p4 ist ein const Zeiger auf einen const int 
	\end{tabular}

        \subsubsection{Shared Pointers}
        Pointertyp, der sich die Anzahl an Shared Pointern, die auf unser Objekt zeigen merkt und es automatisch löscht, sobald kein Pointer mehr darauf zeigt.
        \begin{itemize}
            \item \code{std::shared\_ptr<T> -> } \text{Shared Pointer Klasse auf Objekt von Typ T } 
            \item \code{sharedPtr.use\_count() -> } \text{Anzahl an Pointern, die auf das Objekt zeigt.}
        \end{itemize}

        \subsubsection{Unique Pointers}
        Pointertyp, von dem immer nur genau ein Pointer auf dasselbe Objekt zeigen darf und das objekt automatisch löscht, sobald der Unique Pointer out-of-scope geht.
        \begin{itemize}
            \item \code{std::unique\_ptr<T> -> } \text{Unique Pointer Klasse auf Objekt von Typ T } 
            \item \code{oldUniquePtr.move(newUniquePtr) -> } \text{Anzahl an Pointern, die auf das Objekt zeigt.}
        \end{itemize}
	
	\subsection{Verkettete Listen}
	Kein zusammenhängender Speicherbereich sondern jedes element zeigt auf seinen Nachfolger.
	\subsubsection{Realisierung}
	\code{\textcolor{red}{struct} llnode\{ \\
	\phantom{struct} \textcolor{red}{int} value; \\
	\phantom{struct} llnode* next; \\
	\\
	\phantom{struct} llnode(\textcolor{red}{int} v, llnode* n) : value(v) next(n) \{\} \textcolor{blue}{$\backslash$$\backslash$ Constructor} \\
	\};}
	
	\subsubsection{Vektor durch Linked List}
	Entspricht einem Zeiger auf das erste Element.\\
	\code{\textcolor{red}{class} llvec \{ \\
	\phantom{class }llnode* head; \\
	\textcolor{red}{public}: \\
	\phantom{class }llvec(\textcolor{red}{unsigned int} size); \\
	\phantom{class }\textcolor{red}{unsigned int} size() \textcolor{red}{const}; \\
	\};}
	
	\subsubsection{Brauchbare Funktionen}
	\code{\begin{tabular}{l l}
		print & \textcolor{red}{void} llvec::print(std::ostream\& sink) \textcolor{red}{const} \{ \\
		& \phantom{void }\textcolor{red}{for} (llnode* n = \textcolor{red}{this}-$>$head; n != \textcolor{red}{nullptr}; n = n->next) \{ \\
		& \phantom{void for }sink << n->value << ' '; \}\} \\
		\\
		operator[] & \textcolor{red}{int\&} llvec::\textcolor{red}{operator}[](\textcolor{red}{unsigned int} i) \{ \\
		& \phantom{int6 }llnode* n = \textcolor{red}{this}->head; \\
		& \phantom{int6 }\textcolor{red}{for} ( ; 0 $<$ i; --i) n = n->next; \\
		& \phantom{int6 }\textcolor{red}{return} n->value; \} \\
		\\
		push$\_$front & \textcolor{red}{void} llvec::push$\_$front(\textcolor{red}{int} e) \{ \\
		& \phantom{void }\textcolor{red}{this}->head = \textcolor{red}{new} llnode\{e, \textcolor{red}{this}->head\}; \} \\
		\\
	\end{tabular}
	}
	
	
	\subsection{Speicherverwaltung}
	
	\subsubsection{Der Delete-Ausdruck}
	\code{\textcolor{red}{delete} p;}
	\begin{itemize}
		\item p ist ein Zeiger, der auf ein vorher mit new erzeugtes Objekt zeigt
	\end{itemize}
	\code{\textcolor{red}{delete}[] p;}
	\begin{itemize}
		\item p ist ein Zeiger, der auf ein vorher mit new erzeugtes Array zeigt.
	\end{itemize}
	
	\subsubsection{Der Destruktor}
	\begin{itemize}
		\item Deklaration: \code{$\backsim$Type()}
		\item Wird automatisch aufgerufen bei Aufruf von delete oder wenn Gültigkeitsbereich endet.
		\item Wird automatisch erzeugt wenn keiner definiert wird
	\end{itemize}
	
	\subsubsection{Die Dreierregel}
	\begin{tabular}{l l}
		\textbf{Destruktor} & Löscht Elemente \\
		\textbf{Copy-Konstruktor} & Deklaration: \code{Type(const Type\& x)}\\
		& Für korrektes Kopieren und Initialisieren\\
		\textbf{Zuweisungsoperator} & \code{operator=} \\
		& Wie Copy-Konstruktor, aber früherer "`Müll"' wird aufgeräumt\\
	\end{tabular}
	
	\subsection{Bäume}
	Sind verallgemeinerte Listen: Knoten können mehrere Nachfolger haben.\\
	\code{\textcolor{red}{struct} tnode \{ \\
	\phantom{struct }\textcolor{red}{char} op; \textcolor{red}{double} val; \\
	\phantom{struct }tnode* left; tnode* right;\\
	\\
	tnode(\textcolor{red}{char} o, \textcolor{red}{double} v, tnode* l, tnode* r) : op(o), val(v), left(l), right(r) \{\};
	}
	
	\subsubsection{Brauchbare Funktionen}
	\code{\textcolor{blue}{$\backslash$$\backslash$ POST: returns the size (number of nodes) of the subtree with root n}\\
	\textcolor{red}{int} size (\textcolor{red}{const} tnode* n) \{ \\
	\phantom{int }\textcolor{red}{if} (n)\{\\
	\phantom{int if}\textcolor{red}{return} size(n->left) + size(n->right) + 1;\\
	\phantom{int }\}\\
	\phantom{int}\textcolor{red}{return} 0;\\
	\}
		}\\
	
	\code{\textcolor{blue}{$\backslash$$\backslash$ POST: evaluates the subtree with root n}\\
	\textcolor{red}{double} eval(\textcolor{red}{const} tnode* n) \{ \\
	\phantom{double }assert(n);\\
	\phantom{double }\textcolor{red}{if} (n->op == '=') \textcolor{red}{return} n->val; \textcolor{blue}{$\backslash$$\backslash$ Blatt}\\
	\phantom{double }\textcolor{red}{double} l = 0;\\
	\phantom{double }\textcolor{red}{if} (n->left)) l = eval(n->left); \textcolor{blue}{$\backslash$$\backslash$ linker und rechter Ast}\\ 
	\phantom{double }\textcolor{red}{double} r = eval(n->right);\\
	\phantom{double }\textcolor{red}{switch}(n->op)\{\\
	\phantom{double switch}\textcolor{red}{case} '+': \textcolor{red}{return} l+r;\\
	\phantom{double switch}\textcolor{red}{case} '-': \textcolor{red}{return} l-r;\\
	\phantom{double switch}\textcolor{red}{case} '*': \textcolor{red}{return} l*r;\\
	\phantom{double switch}\textcolor{red}{case} '/': \textcolor{red}{return} l/r;\\
	\phantom{double switch}\textcolor{red}{default:} \textcolor{red}{return} 0;\\
	\phantom{double }\}\\
	\}
		}\\
	\code{\begin{tabular}{l l}
		copy & tnode* copy (\textcolor{red}{const} tnode* n) \{\\
		&\phantom{tnode* }\textcolor{red}{if} (n == \textcolor{red}{nullptr})\\
		& \phantom{tnode* if}\textcolor{red}{return} \textcolor{red}{nullptr};\\
		& \phantom{tnode* }\textcolor{red}{return} new tnode (n->op, n->val,\\
		& \phantom{tnode* }copy(n->left), copy(n->right));\\
		&\}\\
		\\
		clear & \textcolor{red}{void} clear(tnode* n) \{ \\
		& \phantom{void }\textcolor{red}{if}(n)\{ \\
		& \phantom{void if}clear(n->left);\\
		& \phantom{void if}clear(n->right);\\
		& \phantom{void if}\textcolor{red}{delete} n;\\
		& \phantom{void }\}\\
		&\}\\
		\\
	\end{tabular}
	}

%------------------------------------------

\section{Containers}
	Ansammlungs-Datenstrukturen.
	
	\subsection{Funktionen}
	\begin{tabular}{l l}
	\textbf{contains(c, e)} & true, wenn Container c Element e enthält \\
	\textbf{min/max(c)} & Gibt das grösste/kleinste Element zurück \\
		\textbf{sort(c)} & Sortiert die Elemente \\
		\textbf{replace(c, e1, e1)} & Ersetzt alle e1 in c durch e2 \\
		\textbf{sample(c, n)} & Wählt zufällig n elemente aus c aus
	\end{tabular}
	
	\subsection{Iteratoren}
	\begin{tabular}{l l}
	\textbf{it = c.begin()} & Iterator aufs erste Element \\
	\textbf{it = c.end()} & Iterator hinters letzte Element \\
	\textbf{*it} & Zugriff aufs aktuelle Element \\
	++\textbf{it} & Iterator um ein Element verschieben
		
	\end{tabular}
	
	\subsubsection{Const-Iteratoren}
	\vspace{0.1cm}
	\code{cname::const\_iterator it = c.cbegin();}\\
	Gestatten Lesezugriff bei konstanten Containern.
	\subsection{Beispiele}
	\begin{tabular}{l l}
		\textbf{std::unordered$\_$set$<$Type$>$} & Ungeordnete, duplikatfreie  \\
		& Zusammenfassung von Elementen. \\
		\textbf{std::set$<$Type$>$} & Geordnete, duplikatfreie Zusammenfassung \\
		& von Elementen. \\
	\end{tabular}


%--------------------------------------------

\section{Extras}

	\subsection{Assertions}
	\code{assert(expr);}
	\begin{itemize}
		\item benötigt \code{\#include $<$cassert$>$}
		\item Programm wird beendet falls expr nicht true
		\item abschalten durch \code{\#define NDEBUG}
	\end{itemize}
	
	\subsection{Typ-Aliasse}
	\code{using Name = Typ \textcolor{blue}{$\backslash$$\backslash$Typ kann neu mit Name angesprochen werden}
	}
	\subsection{Die Standardbibliothek}
	benötigt \code{\#include} \\
	\begin{tabular}{l l l}
	\code{<iostream>} & \code{std::cin>>} \\
	& \code{std::cout<<} \\
	\hline
	\code{<cmath>} & \code{std::pow()} & Potenzieren \\
	& \code{std::sqrt()} & Quadratwurzel \\
	& \code{std::abs()} & Absolutbetrag \\
	\hline
	\code{<algorithm>} & \code{std::max} & Maximum zweier Argumente
	\end{tabular}
	
	
	\subsection{Fehlerquellen}
	\vspace{0.1cm}
	\begin{itemize}
		\item Deklarationen sind nur im Block gültig.
		\item Funktionsargumente nicht vergessen.
		\item Wenn man eine Referenz erzeugt, muss das Objekt, auf das sie verweist, mindestens so lange leben wie die Referenz selbst.
		\item Der Zugriff auf Elemente ausserhalb der gültigen Grenzen eines Vektors führt zu undefiniertem Verhalten.
		\item Verhalten von Funktionen ungleich void ist undefiniert, wenn das Ende des Rumpfes ohne return-Anweisung erreicht wird.
		\item Vergleichsoperatoren existieren für structs und classes per default nicht.
	\end{itemize}
	
	


												
    \end{multicols*}
\setcounter{secnumdepth}{2}
\end{document}
